%%%%%%%%%%%%%%%%%%%%%%%%%%%%%%%%%%%%%%%%%%%%%%%%%%%%%%%%%%%%%%%%%%%%%%%%%%%%%
%%%
%%% File: utthesis2.doc, version 2.0jab, February 2002
%%%
%%% Based on: utthesis.doc, version 2.0, January 1995
%%% =============================================
%%% Copyright (c) 1995 by Dinesh Das.  All rights reserved.
%%% This file is free and can be modified or distributed as long as
%%% you meet the following conditions:
%%%
%%% (1) This copyright notice is kept intact on all modified copies.
%%% (2) If you modify this file, you MUST NOT use the original file name.
%%%
%%% This file contains a template that can be used with the package
%%% utthesis.sty and LaTeX2e to produce a thesis that meets the requirements
%%% of the Graduate School of The University of Texas at Austin.
%%%
%%% All of the commands defined by utthesis.sty have default values (see
%%% the file utthesis.sty for these values).  Thus, theoretically, you
%%% don't need to define values for any of them; you can run this file
%%% through LaTeX2e and produce an acceptable thesis, without any text.
%%% However, you probably want to set at least some of the macros (like
%%% \thesisauthor).  In that case, replace "..." with appropriate values,
%%% and uncomment the line (by removing the leading %'s).
%%%
%%%%%%%%%%%%%%%%%%%%%%%%%%%%%%%%%%%%%%%%%%%%%%%%%%%%%%%%%%%%%%%%%%%%%%%%%%%%%

\documentclass[12pt]{report}         %% LaTeX2e document.
                                     %% PR keep 12pt to be within 2020 format specification

%% Preamble starts here and goes until \begin{document}.

%% Special packages, kept (optionally) in their own directory for
%% general housekeeping purposes. Note that the path to the .sty
%% file must be specified if it is kept in a separate directory.
\usepackage{./packages/utthesis2}              
\usepackage{./packages/gb4e}

%% Built-in packages
\usepackage{multicol}   %% Multicolumnar support
\usepackage{graphicx}   %% Support for \includegraphics with scale options
\usepackage{lipsum}     %% Random text generator, just for sample file
\usepackage{nomencl}    %% Nomenclature package
\usepackage{natbib}     %% Bibliography package
\usepackage{glossaries} %% Glossary package
\usepackage{mfirstuc}   %% Capitalize first letters (for chapter/section names)
\usepackage{tocloft}    %% Improved TOC, LOT & LOF handler
\usepackage[linktocpage=true]{hyperref}   %% Clickable hyperlinks pakage

%%% tocloft configuration for TOC, LOT and LOF
%% PR this is a new configuration for using tocloft to generate relevant pages

\renewcommand{\contentsname}{Table of Contents\hfill}   
%% TOC, LOT & LOF titles positioning
\renewcommand{\cftbeforetoctitleskip}{0pt}
\renewcommand{\cftaftertoctitleskip}{10pt}
\renewcommand{\cftbeforelottitleskip}{0pt}
\renewcommand{\cftafterlottitleskip}{10pt}
\renewcommand{\cftbeforeloftitleskip}{0pt}
\renewcommand{\cftafterloftitleskip}{10pt}
%% TOC, LOT & LOF titles positioning
\renewcommand{\cfttoctitlefont}{\hfill\bfseries\large}
\renewcommand{\cftaftertoctitle}{\hfill}
\renewcommand{\cftlottitlefont}{\hfill\bfseries\large}
\renewcommand{\cftafterlottitle}{\hfill}
\renewcommand{\cftloftitlefont}{\hfill\bfseries\large}
\renewcommand{\cftafterloftitle}{\hfill}

%% TOC vertical spacing
\renewcommand\cftbeforepartskip{10pt}
\renewcommand\cftbeforechapskip{5pt}
\renewcommand\cftbeforesecskip{2pt}
\renewcommand\cftbeforesubsecskip{2pt}

%TOC font sizes
\renewcommand\cftpartfont{\large\bfseries}
\renewcommand\cftchapfont{\normalsize}
\renewcommand\cftsecfont{\normalsize}
\renewcommand\cftchappagefont{\normalsize}
\renewcommand\cftsecpagefont{\normalsize}
%% Other font sizes
\renewcommand{\cftfigfont}{\normalsize}     % figure entries in LOT
\renewcommand{\cftfigpagefont}{\normalsize} % figure caption pages
%% TOC dots
\renewcommand{\cftpartleader}{\cftdotfill{\cftdotsep}} % dots for parts
\renewcommand{\cftchapleader}{\cftdotfill{\cftdotsep}} % dots for chapters

%% PR setting hyperlink colors black
%% Hyperlink colors
\hypersetup{
    colorlinks,
    citecolor=black,
    filecolor=black,
    linkcolor=black,
    urlcolor=black
}



\newif\ifmasters		     %% Do not change
\newif\ifphd			     %% these lines.

% \mastersthesis   \masterstrue      %% Uncomment one of these lines.
 \phdthesis      \phdtrue           %% 


% \leftchapter                       %% Uncomment one of these if you want
 \centerchapter                     %% left-justified, centered or
% \rightchapter                      %% right-justified chapter headings.
                                     %% Chapter headings includes the
                                     %% Contents, Acknowledgments, Lists
                                     %% of Tables and Figures and the Vita.
                                     %% The default is \centerchapter.

% \singlespace                       %% Uncomment one of these if you want
\oneandhalfspace		     %% single-spacing, space-and-a-half
% \doublespace                       %% or double-spacing; the default is
                                     %% \oneandhalfspace, which is the
                                     %% minimum spacing accepted by the
                                     %% Graduate School.

%% Your official UT name.
\renewcommand{\thesisauthor}{(Insert your Official UT Name)}    

%% Your month of graduation.
\renewcommand{\thesismonth}{May}     

%% Your year of graduation.
\renewcommand{\thesisyear}{2020}      

%% The title of your thesis; use mixed-case.
\renewcommand{\thesistitle}{A Quick Guide to Starting your 
			    UT Thesis with \LaTeX  \\ second line }
				     
%% Your previous degrees, abbreviated; separate multiple degrees by commas.     
\renewcommand{\thesisauthorpreviousdegrees}{(list your previous degrees here)}

%%%%%%%%%%%%%%%%%%%%%%%%%%%%%%%%%%%%%%
%%% Thesis Committee Member Section:
%%% 
%%% For both Master's and PhD, you must declare \thesissupervisor
%%% For Master's, enter your second reader as \thesiscommitteemembera,
%%% found at the top of the block right past where you can define your PhD
%%% co-supervisor. \thesissupervisor and \thesiscommittemembera are the only
%%% fields used in creating Master's theses and reports.
                                     
%% Your thesis supervisor; use mixed-case and don't use any titles or degrees.
\renewcommand{\thesissupervisor}{Ada Lovelace}

\newif\ifcosuper                                     
%% Your PhD. thesis co-supervisor; if any. Use mixed case and don't use any 
%% titles or degrees. Uncomment if you have a co-supervisor. Do not change
%% the following declaration \cosupertrue. (Ignored for Master's)
% \renewcommand{\thesiscosupervisor}{} \cosupertrue

%% Your remaining committee members. Only the first of these is used for
%% Master's theses/reports -- rest are applicable to Ph.D. only, ignored for
%% Master's
 \renewcommand{\thesiscommitteemembera}{James Bond}
 \renewcommand{\thesiscommitteememberb}{Professor 3}
 \renewcommand{\thesiscommitteememberc}{Professor 4}
 \renewcommand{\thesiscommitteememberd}{Professor 5}
% \renewcommand{\thesiscommitteemembere}{}
% \renewcommand{\thesiscommitteememberf}{}
% \renewcommand{\thesiscommitteememberg}{}

%% Your permanent address; use "\\" for linebreaks.
\renewcommand{\thesisauthoraddress}{}

%% Your dedication, if you have one; use "\\" for linebreaks.
\renewcommand{\thesisdedication}{Put your dedication here.}

%%%%%%%%%%%%%%%%%%%%%%%%%%%%%%%%%%%%%%%%%%%%%%%%%%%%%%%%%%%%%%%%%%%%%%%%%%%%%
%%%
%%% The following commands are all optional, but useful if your requirements
%%% are different from the default values in utthesis.sty.  To use them,
%%% simply uncomment them.


%\renewcommand{\thesiscommitteesize}{...}
%\renewcommand{\masterscommitteeminusone}{...} %% This line not needed for Ph.D.
				    %% Uncomment these only if your thesis
				    %% committee does NOT have 5 members
				    %% for \phdthesis or 2 for \mastersthesis.
				    %% Replace the "..." with the correct
				    %% number of members. 

% \renewcommand{\thesisdegree}{...}   
				    %% Uncomment this only if your thesis
				    %% degree is NOT "DOCTOR OF PHILOSOPHY"
				    %% for \phdthesis or "MASTER OF ARTS"
				    %% for \mastersthesis.  Provide the
				    %% correct FULL OFFICIAL name of
				    %% the degree.

% \renewcommand{\thesisdegreeabbreviation}{...}
                                    %% Use this if you also use the above
                                    %% command; provide the OFFICIAL
                                    %% abbreviation of your thesis degree.

% \renewcommand{\thesistype}{...}    
				    %% Use this ONLY if your thesis type
                                    %% is NOT "Dissertation" for \phdthesis
                                    %% or "Thesis" for \mastersthesis.
                                    %% Provide the OFFICIAL type of the
                                    %% thesis; use mixed-case.

% \renewcommand{\thesistypist}{...} 
				    %% Use this to specify the name of
                                    %% the thesis typist if it is anything
                                    %% other than "the author".

%% Use these commands if you want to change defaults for section numbering
%% and inclusion in your table of contents
% \setcounter{secnumdepth}{3}
% \setcounter{tocdepth}{3}

%%%%%%%%%%%%%%%%%%%%%%%%%%%%%%%%%%%%%%%%%%%%%%%%%%%%%%%%%%%%%%%%%%%%%%%%%%%%%

\makenomenclature
\makeglossaries
\input{glossary}


\begin{document}

\thesiscopyrightpage                 %% Generate the copyright page.

\thesiscertificationpage	     %% Generate the certification page.
				     
\thesistitlepage                     %% Generate the title page.

\thesisdedicationpage                %% Generate the dedication page.

%% Use this to write your acknowledgments; it can be anything
%% allowed in LaTeX2e par-mode. This can be input as a separate .tex
%% file or typed directly here. If typed here, remove the \input line
%% and type your text in its place.
\begin{thesisacknowledgments}        
Put your acknowledgments here.
\lipsum[1-4]
              
\end{thesisacknowledgments}          

%% Use this to write your thesis abstract; it can be anything
%% allowed in LaTeX2e par-mode. This can be input as a separate .tex
%% file or typed directly here. If typed here, remove the \input line
%% and type your text in its place.
\begin{thesisabstract}               
Put your abstract here. Should not exceed 350 words. It should be a continuous description, not disconnected notes or an outline

\lipsum[1-3]
                    
\end{thesisabstract}                 

\tableofcontents                     %% Generate table of contents.
\cleardoublepage
\addcontentsline{toc}{chapter}{\texorpdfstring{\capitalisewords{List of tables}}{}}
\listoftables         %% Uncomment this to generate list of tables.

\cleardoublepage
\addcontentsline{toc}{chapter}{\texorpdfstring{\capitalisewords{List of figures}}{}}
\listoffigures       %% Uncomment this to generate list of figures.                                    


\input{nomenclature}			     %% Input as separate file (recommended)
\printnomenclature
\addcontentsline{toc}{chapter}{\texorpdfstring{\capitalisewords{Nomenclature}}{}}

%% PR New functionality of major sections (hierarchy above chapters)
\part*{\MakeUppercase{First major section of this document}} 
\addcontentsline{toc}{part}{\texorpdfstring{\MakeUppercase{First major section of this document}}{}}

%\nonewpagechapter{\capitalisewords{Introduction}} %% Create a chapter.
\nonewpagechapter{Introduction}
                             %% PR if comes after new major section uses a no-new-page 
\label{chap:intro}		     %% Label, if desired, for referencing.
\input{intro}			     %% Input as separate file (recommended)

%% PR following is to provide more examples

\chapter{\capitalisewords{Literature review}}
\label{chap:lit_rev}
\lipsum[1-3]

\part*{Second part of this document}
\addcontentsline{toc}{part}{\texorpdfstring{\MakeUppercase{Second Part of this document}}{}}

\lipsum[1]

\nonewpagechapter{\capitalisewords{Background}}
\label{chap:background}
This chapter was included just to have another chapter, nothing to see here.

The \Gls{latex} typesetting markup language is specially suitable 
for documents that include \gls{maths}.

%% If you have appendices, uncomment the following line and include
%% your appendices below. 
%% NG: This is a bit of a workaround to accommodate recent formatting
%% changes whereby appendices need to be formatted like chapters, but
%% listed in the ToC like sections. That's the kind of incongruity that
%% LaTeX really hates so the unfortunate side effects are: You have to
%% manually label your appendix titles and ToC entries, and arguably worse,
%% it breaks the ability to dynamically cross-reference your appendices.
%% So if you mention "blah blah Appendix \ref{chap:appendix}", you'll get
%% the wrong thing here. Could be worse, but manually fixing references
%% to appendix titles in your final version is still an annoyance and
%% potential gotcha.
\appendix

%% Subsequent appendices will be included as below, EXCEPT remove the
%% chapter-level \addcontentsline statement.
\chapter*{Appendix A\\Sample Appendix}
\addcontentsline{toc}{chapter}{Appendices}
\addcontentsline{toc}{section}{A: Sample Appendix}
\label{chap:appendix}
\input{appendix}
 
%% PR from formatting guide:
%%This page is optional—must be placed in this order if it is included in the dissertation.
%If you don’t want to include a glossary, then delete the entire page and the following page break.
\printglossaries
\addcontentsline {toc}{chapter}{Glossary} 



%% Generate bibliography from your .bib file, in the style of your choice.
\bibliography{mybib}                    
%% Force Bibliography to appear in contents                       
\addcontentsline {toc}{chapter}{Bibliography} 
%\bibliographystyle{./packages/natbib.sty} %% custom style in packaged is incompatible
\bibliographystyle{plain}

%% Alternatively, comment out the two preceding lines and uncomment the two
%% following lines if you want to enter your bibliography items directly here.
% \begin{thebibliography}{..} 
% \end{thebibliography}                

%% Write your vita here; it can be anything allowed in in LaTeX2e par-mode.
%% Including a vita, which lists your permanent address, may be optional.
\begin{thesisauthorvita}    

This page is optional:  if you do not include a Vita, delete this entire page.  The vita is a brief biographical sketch of the writer that provides information for future readers.  It should include the writer's full name and a permanent address or email address where the author can be reached. Because your dissertation will be available electronically, be aware that certain personal information could be used to steal your identity. For this reason, you are advised not to include your date of birth, parents’ names, or children’s names. The name of the typist should always appear at the end of the page

\end{thesisauthorvita}               

\end{document}                       %% Done.
